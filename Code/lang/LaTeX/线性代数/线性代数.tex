%!TEX program = xelatex
% This file can be compiled by XeLaTeX

%% Head %%

\documentclass[11pt]{ctexart}
% \documentclass[UTF8]{ctexart}
\usepackage[b5paper,left=1cm,right=1cm,top=2cm,bottom=2cm]{geometry}
\usepackage{xeCJK}
  % \setCJKmainfont{Noto Sans CJK SC}
\usepackage{ctex}     % Chinese TeX
\usepackage{ulem}       % Underlines or Crossout
\usepackage{amsmath}
\usepackage{mathdots}

\setmainfont{Noto Sans}
\parskip = 0.8em plus 1pt
\parindent = 0pt

%% Body %%

\begin{document}

\title{线性代数}
\author{Jeff F.}
\maketitle

\section{行列式}

\subsection{基本概念}

\newcommand{\bdarrow}{\text{\fontspec{Noto Sans CJK SC Regular} ⇔ }}
\begin{itemize}
\item 逆序数
\item $det(a_{ij}) ~ \bdarrow ~D~ \bdarrow \text{某行列式}$
\end{itemize}

\subsection{基本计算}

三阶对角线法则:

\begin{align*}
&
\begin{vmatrix}
    a_{11} & a_{12} & a_{13} \\
    a_{21} & a_{22} & a_{23} \\
    a_{31} & a_{32} & a_{33}
\end{vmatrix} \\
 = \quad &a_{11}a_{22}a_{33} + a_{12}a_{23}a_{31} + a_{13}a_{21}a_{32} \\
       - &a_{11}a_{23}a_{32} - a_{12}a_{21}a_{33} - a_{13}a_{22}a_{31}
\end{align*}

n 阶行列式计算
\[
D = \begin{vmatrix}
    a_{11} & a_{12} & \cdots & a_{1n} \\
    a_{21} & a_{22} & \cdots & a_{2n} \\
    \vdots & \vdots & \ddots & \vdots \\
    a_{n1} & a_{n2} & \cdots & a_{nn}
\end{vmatrix}
=
\sum (-1)^t a_{1p_{1}} a_{2p_{2}} \dots a_{np_{n}}
\]
\emph{t为逆序数}

% \begin{table}
% \centering %necessary in order for the table to be centered
% \caption{This is a LaTeX Table}
% \begin{tabular}{lcr}
% First  & $x^2+y^2$ & $\frac{a}{b}$ \\
% Second & 0 & Table Cell \\
% Third  & $y=\sqrt{(1-x^2)}$ & End
% \end{tabular}
% \end{table}

对于二元线性方程组:
\[
\begin{cases}
    a_{11}x_1 + a_{12}x_2 = b_1,\\
    a_{21}x_1 + a_{22}x_2 = b_2.\\
\end{cases}
\]
\[
x_1 = \frac{D_1}{D} =
    \frac{
        \begin{vmatrix}
            b_{1} & a_{12} \\
            b_{2} & a_{22}
        \end{vmatrix}
    }{
        \begin{vmatrix}
            a_{11} & a_{12} \\
            a_{21} & a_{22}
        \end{vmatrix}
    }
,\quad
x_2 = \frac{D_2}{D} =
    \frac{
        \begin{vmatrix}
            a_{11} & b_{1} \\
            a_{21} & b_{2}
        \end{vmatrix}
    }{
        \begin{vmatrix}
            a_{11} & a_{12} \\
            a_{21} & a_{22}
        \end{vmatrix}
    }
\]

\section{矩阵}
\section{线性相关性}
\section{二次型}
\section{线性空间与线性变换}


\end{document}
